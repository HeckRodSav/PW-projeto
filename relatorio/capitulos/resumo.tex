% \begin{abstract}
%   This meta-paper describes the style to be used in articles and short papers
%   for SBC conferences. For papers in English, you should add just an abstract
%   while for the papers in Portuguese, we also ask for an abstract in
%   Portuguese (``resumo''). In both cases, abstracts should not have more than
%   10 lines and must be in the first page of the paper.
% \end{abstract}
     
\begin{resumo} 
  Atualmente no contexto nacional somente cerca de 30\% da população conta com um plano de saúde particular \cite{brunobocchini2018}, e muitas vezes, cerca de 40\% dos casos, o paciente brasileiro prefere fazer um autodiagnóstico baseado em informações retiradas da internet, que pode ser danoso à sua saúde \cite{vanessathees2018}.
  No presente trabalho é proposto um sistema online para diagnósticos simples, através de um linguajar menos técnico, e sempre deixando claro que procurar um médico antes de se automedicar é altamente recomendado.
  As bases de dados mais encontradas online trabalham com organização top down, listando doenças e descrevendo seu sintomas, fazendo até que o paciente desenvolva os demais sintomas da doença de forma psicossomática \cite{contaifercavalcante2018}.
  O diferencial desta proposta é a utilização de uma abordagem bottom up, obtendo e cruzando dados, através de questões a respeito de hábitos e sintomas do paciente, utilizando como amparo sistema de auxílio à tomada de decisão.
\end{resumo}