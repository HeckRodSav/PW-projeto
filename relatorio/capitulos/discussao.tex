\section{Discussão}

No que diz respeito a contribuição à sociedade, o sistema se prontifica a organizar os questionários de forma objetiva, fazendo o possível para não sugerir erroneamente sintomas que o paciente não tenha apresentado, sendo assim, evitando o surgimento psicossomático destes.
A interface conta com botões grandes e cores calmas, melhorando a experiência do usuário.

As maiores dificuldades enfrentadas foram relacionadas a construção da base de dados para o sistema e a interação do banco de dados completo para com a aplicação.
A melhor base de dados encontrada estava em alemão \cite{leander}, fazendo-se necessária a refatoração geral, com tradução para o português, além da adaptação geral para submeter os dados no banco escolhido.

Enquanto completava-se o banco, também era necessário interagir corretamente com os dados por parte da aplicação. A construção de lógica para selecionar a próxima pergunta do sistema foi um pequeno desafio em primeira instância, mas logo resolvido.