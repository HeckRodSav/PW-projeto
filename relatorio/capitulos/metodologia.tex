\section{Metodologia}

A metodologia escolhida para a aplicação consiste num modelo \emph{Model View Controler} (MVC), onde cada camada será responsável por uma função, sendo estas:
\begin{itemize}
    \item \emph{Views}: tudo que diz respeito a interação com usuário, ou o questionário de sintomas.
    \item \emph{Controlers}: serão responsáveis pela comunicação com os dados sobre doenças e pela aplicação de lógica sobre os sintomas apresentados pelo usuário, também como quem devolverá os possíveis passos para prevenção e tratamento para o diagnóstico, nunca deixando de reforçar a necessidade de buscar ajuda profissional.
    \item \emph{Models}: serão as entidades que representarão os dados usado pela aplicação, ou seja, as doenças, sintomas, tratamentos e talvez algum dado ainda não previsto. Ainda fará, através de classes repositórios, a persistência dos dados no banco e consultas ao mesmo.
\end{itemize}


Temos como referência de aplicação o Guia de Doenças e Sintomas \cite{alberteinstein}, tanto como possível base de dados a ser compilados, quanto em como forma de levar o questionário ao usuário.
Por outro lado, também ilustra o cenário cujo não desejamos ser vítimas, o de receber uma lista assustadora de doenças relacionadas ao seus sintomas, tais como câncer ou insuficiência renal, quando estão ligadas ao caso por poucos sintomas.


Como citado antes, a estrutura da aplicação terá 3 camadas, cada uma sendo devidamente encapsulada e fazendo a comunicação com a outra de forma segura e sem interferir no contexto.

Iniciando pelo banco de dados, consistirá em  doenças e seus tratamentos e sintomas relacionados as mesmas, sendo que pretendemos definir um valor de probabilidade para esse relacionamento, para que tenha o papel de índice de tomada de decisão para o quão provável   que o usuário tenha a doença baseado neste número. Como esse ÍNDICE DE INCIDÊNCIA será calculado ainda não está claro.

Em seguida vem a camada de controlers, onde teremos a interação do banco com as informações fornecidas pelo usuário, ou seja, questões simples sobre os sintomas registrados no banco serão enviadas para o front, a resposta voltará para a controler e baseado nisso, iremos montar um mapeamento, excluindo as doenças que não incluem as afirmativas de sintomas e incluindo as que sim, até que cheguemos numa possível resposta, devolvendo para o usuário o diagnóstico e como prevenir e tratar.
Sempre lembrando que essa avaliação de qual diagnóstico adotar, será baseada nesse índice de incidência que está em desenvolvimento e avaliação para melhor aproveitamento.


Por fim, teremos o front, ou a interface vista pelo usuário. Será estruturada como um quiz, que apresentará questões de sim ou não para os sintomas, até que cominará no diagnóstico possível.


