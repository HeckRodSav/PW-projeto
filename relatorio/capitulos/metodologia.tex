\section{Metodologia}

A metodologia escolhida para a aplicação consiste num modelo \emph{Model View Controler} (MVC), onde cada camada será responsável por uma função, sendo estas:
\begin{itemize}
    \item \emph{Views}: tudo que diz respeito a interação com usuário, ou o questionário de sintomas.
    \item \emph{Controlers}: serão responsáveis pela comunicação com os dados sobre doenças e pela aplicação de lógica sobre os sintomas apresentados pelo usuário, também como quem devolverá os possíveis passos para prevenção e tratamento para o diagnóstico, nunca deixando de reforçar a necessidade de buscar ajuda profissional.
    \item \emph{Models}: serão as entidades que representarão os dados usado pela aplicação, ou seja, as doenças, sintomas, tratamentos e talvez algum dado ainda não previsto. Ainda fará, através de classes repositórios, a persistência dos dados no banco e consultas ao mesmo.
\end{itemize}


Temos como referência de aplicação o Guia de Doenças e Sintomas \cite{alberteinstein}, tanto como possível base de dados a ser compilados, quanto em como forma de levar o questionário ao usuário.
Por outro lado, também ilustra o cenário cujo não desejamos ser vítimas, o de receber uma lista assustadora de doenças relacionadas ao seus sintomas, tais como câncer ou insuficiência renal, quando estão ligadas ao caso por poucos sintomas.


