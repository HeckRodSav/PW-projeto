\section{Metodologia}

\subsection{Componentes}

A metodologia escolhida para a aplicação consiste num modelo \emph{Model View Controller} (MVC), onde cada camada será responsável por uma função, sendo estas:
\begin{itemize}
    \item \emph{Views}: tudo que diz respeito a interação com usuário, ou o questionário de sintomas.
    \item \emph{Controllers}: serão responsáveis pela comunicação com os dados sobre doenças e pela aplicação de lógica sobre os sintomas apresentados pelo usuário, também como quem devolverá os possíveis passos para prevenção e tratamento para o diagnóstico, nunca deixando de reforçar a necessidade de buscar ajuda profissional.
    \item \emph{Models}: serão as entidades que representarão os dados usado pela aplicação, ou seja, as doenças, sintomas, tratamentos e talvez algum dado ainda não previsto. Ainda fará, através de classes repositórios, a persistência dos dados no banco e consultas ao mesmo.
\end{itemize}

Tais camadas sendo devidamente encapsulada e fazendo a comunicação com as outras de forma segura e sem interferir no contexto.

\subsubsection{Views}

Também definido como \emph{front}, ou a interface vista pelo usuário. Será estruturada como um quiz, que apresentará questões booleanas, de verdadeiro ou falso, a respeito dos sintomas do paciente.
As informações obtidas neste formulário serão enviadas para a camada \emph{Controllers}.

\subsubsection{Controllers}

Aqui ocorre a interação do banco com as informações fornecidas pelo usuário, ou seja, questões simples sobre os sintomas registrados no banco serão enviadas para o \emph{front}, em seguida, a resposta voltará para a \emph{controller}.

Baseado nisso, será realizado um mapeamento, excluindo doença que não incluem os sintomas indicados e incluindo as demais, até que o sistema aponte uma possível resposta, devolvendo para o usuário o diagnóstico, uma possível prevenção e alguma forma de tratamento.
Sempre lembrando que a avaliação de qual diagnóstico adotar será baseada num índice de incidência que está em desenvolvimento e avaliação para melhor aproveitamento, citado na subseção \ref{ssec:models}.

\subsubsection{Models}\label{ssec:models}

Consiste no banco de dados, relacionando doenças a seus sintomas e alguns possíveis tratamentos.
Pretendemos definir um valor de probabilidade para esse relacionamento, para que tenha o papel de índice de tomada de decisão para a probabilidade do usuário apresentar a doença em questão baseado neste número.
Ainda não temos plena certeza de como calcular este valor de incidência, porém algumas bibliografias apresentam dados relevantes\cite{AlbertEinstein, longo2011harrison}.

\subsection{Opções similares}

Temos como referência de aplicação o Guia de Doenças e Sintomas \cite{AlbertEinstein}, tanto como possível base de dados a ser compilados, quanto em como forma de levar o questionário ao usuário.

Apresentando informações completas, é uma excelente ferramenta para sua proposta.
Por outro lado, também ilustra o cenário que desejamos evitar, quando o usuário recebe uma lista assustadora de doenças relacionadas ao seus sintomas, tais como câncer ou insuficiência renal, quando estão ligadas ao caso por poucos sintomas.

\subsection{Avaliação e conjunto de dados}

Avaliação: A avaliação da aplicação seria feita de forma ideal recebendo respostas de um paciente e validando o diagnóstico com um médico capacitado. Embora seja possível mensurar a efetividade com casos de teste, se escolhem doenças e inserindo respostas relacionadas ao sintomas da mesma, não deixando de conferir casos excepsionais, ou  em outras palavras, casos em que o usuário possa estar com sintomas divergentes.
Podem haver testes com usuários voluntários e até opiniões médicas.

\subsection{Complementação}

Respondendo aos questionamentos deixados pelo professor, não utilizaremos sistemas de análise de linguagem natural, já que a entrada do usuário será booleana, como sim ou não.
Não encontramos nenhum módulo disponível para este sistema, mas com uma análise estatística é possível criar uma base com margem de acerto aceeitável.