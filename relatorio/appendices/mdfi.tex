\section{Prototipação de média fidelidade}

\subsection{Dificuldades e simplificações na interface front-end}

Ao longo do desenvolvimento das páginas front-end, algumas dificuldades foram encontradas, bem como detalhes que poderiam ser simplificados.

\subsubsection{Dificuldades encontradas}

\paragraph{Barra lateral}
A principal dificuldade encontrada foi relacionada a barra lateral, empenhando horas de pesquisa de um método sobre como colapsar a barra lateral de maneira mais natural ao usuário conforme o modelo mais adotado atualmente: a barra lateral corre horizontalmente.

A solução encontrada foi simplesmente deixar a barra lateral correndo verticalmente, porém sua animação ficou limitada a altura mínima dos itens em seu interior.

\paragraph{Fundo escurecido}

Na proposta de baixa fidelidade, apêndice \ref{apdx:lofi}, é possível notar que em algumas páginas, quando um item tomasse a frente da interface, os itens ao fundo seriam escurecidos.
A dificuldade envolvida foi relacionada a adequar o elemento HTML ao tamanho lateral da página, devido à margem horizontal pré estabelecida pelo Bootstrap 4.

A solução foi encontrada numa página do StackOveflow, indicando esta característica no elemento com a classe selecionada.

Houve um caso especial onde esta característica foi descartada.
Na versão mobile, quando a barra lateral está visível, o protótipo não apresenta a alteração do fundo.

\subsubsection{Simplificações}

\paragraph{Questionário de sintomas}

Na versão apresentada anteriormente, apêndice \ref{apdx:lofi}, na interface durante o questionário de sintomas, figuras \ref{fig:mobile:symptom} e \ref{fig:desktop:symptom}, as diferentes opções da interface são apresentadas como telas separadas.
Na implementação do protótipo atual, estas opções fazem parte da mesma página.

\paragraph{Questionário sobre sexo biológico}

Bem como no questionário de sintomas, a pergunta sobre sexo biológico, figuras \ref{fig:mobile:bio_sex} e \ref{fig:desktop:bio_sex}, ganha a mesma propriedade visual para o caso de dois botões.

\paragraph{Botão de continuar}

Para cada interface com botão de continuar, este foi apropriadamente bloqueado enquanto uma resposta coerente não fosse dada.

\subsection{Video de demonstração}\label{subsec:demo}

Foram gravados dois vídeos para demonstração do sistema.

\subsubsection{Versão Desktop}

\href{https://youtu.be/cfdYXQWT3AI}{\emph{Link} para vídeo demonstrativo da versão Desktop.}

\subsubsection{Versão Mobile}

\href{https://youtu.be/rhEI6m4BFxc}{\emph{Link} para vídeo demonstrativo da versão Mobile.}

\subsection{Repositório do projeto}

O projeto pode ser encontrado em \href{https://github.com/HeckRodSav/PW-projeto}{<https://github.com/HeckRodSav/PW-projeto>}.
Os arquivos deste versão estão disponível na pasta /aplicacao/front/ do repositório.

A gravação dos vídeos de demonstração, subseção \ref{subsec:demo}, foi feita a partir da página \emph{github.io} do projeto, acessível por \href{https://heckrodsav.github.io/PW-projeto/aplicacao/front/}{<https://heckrodsav.github.io/PW-projeto/aplicacao/front/>}